\chapter*{Aufgabenstellung}

{\large\textbf{CubeSat-based active debris removal }}\\ \\
Viele Bereiche des niedrigen Erdorbits sind bereits stark mit aktiven Satelliten sowie Weltraumschrott jeder Art ausgelastet, und eine weitere Zunahme ist zu erwarten: Mehrere Konstellationen aus Hunderten oder Tausenden Satelliten sind angekündigt, z.B. durch das Unternehmen OneWeb, das plant, bis zu 2620 Satelliten ab 2019 in den Orbit zu bringen. Alle Satelliten dieser "Mega-constellations" werden die Fähigkeit besitzen auf die eine oder andre Weise das Ziel eines zügigen Vergühens in der Erdatmosphäre nach geplanter Betriebsdauer zu erreichen. Bei dieser Anzahl sind allerdings Fehlfunktionen durchaus zu erwarten. Zudem ist die Abwägung zwischen sehr hoher Zuverlässigkeit mit indes gesteigerten Kosten, und Backup-Lösungen wie dem Active Debris Removal (ADR) zu betrachten. Hierbei wird der gestrandete Satellit von einem weiteren, spezialisierten Satelliten gefangen und zum Verglühen gebracht. 
Die meisten ADR Konzepte existieren für das Entfernen einzelner, großer Objekte. Die betrachteten Konstellations-Satelliten sind, mit Massen unter 200 kg, aber eher klein. Damit könnte es möglich und sinnvoll werden, CubeSats für ADR zu nutzen. Diese Arbeit versucht herauszufinden, ob dies eine grundsätzlich sinnvolle Option darstellt.\\ \\
Die Arbeitsschritte beinhalten:
\begin{itemize}
\item Literaturübersicht über Cubesats und Rendezvous und Docking Technologie
\item Einarbeitung in der bisher durchgeführten Arbeiten über Cubesat-based-ADR am IRAS
\item Design einer ADR Mission und den dazugehörigen Cubesats
\item Bewertung der Machbarkeit und Performance der Mission
\end{itemize}
