\chapter{Theoretische Grundlagen}
%\addcontentsline{toc}{chapter}{Grundlagen}
\section{Das Cubesat Designstandard}
	\subsection{Standard Definition}
	\subsection{Historische Entwicklung} kurz
					\textbf{Anwendungsbereich}
					\textbf{Low-Budget LEO Experimente}
					\textbf{Interplanetar (InSight)}
					\textbf{Active Debris Removal}
					\textbf{Cubesat Misisonen}
					Übersicht über bisherige launches und deren payload (Daten und ein Bild zum anpassen existieren schon)
			
	\section{Cubesat Subsysteme}
	hier Hauptsächlich das Fazit von max kompakt darstellen und refernzieren
		\subsection{Antrieb - propulsion}
		\subsection{Energie - EPS}
		\subsection{Guidance, navigation and control -GNC ADCS}
		\subsection{Command and data handling}
		\subsection{Communications}
		\subsection{Thermal}
		\subsection{Structure}
		
	\section{RDVDO Mechanismen} das kann ausführlich sein
		\subsection{Docking Strategien}
						\textbf{Roboterarm}
						\textbf{Fangnetz}
						\textbf{Adhäsiv Docken}
						Übersichtstabelle/Graphik: sihe die Quellen die ich am 15.05.2019 gezeigt habe
		\subsection{Bionische Materialien}
						\textbf{Was sind Geckomaterialen}
						\textbf{Bisher getestete Gecko-Materialien}
						\textbf{Bisherige Erfolge}
						\textbf{State of the Art}
						\textbf{Problematik}