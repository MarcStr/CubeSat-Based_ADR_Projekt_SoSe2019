\chapter{Theoretische Grundlagen}
%\addcontentsline{toc}{chapter}{Grundlagen}
\section{Der Cubesat Designstandard}
	\subsection{Historische Entwicklung}
	\subsection{Gestaltungsrichtlinien}
	
	
	\begin{flushleft}
Für die Gestaltung von CubeSats gelten eine Reihe von Gestaltungsrichtlinien. Als kleinste Einheit (1U) wird ein Würfel mit einer Kantenlänge von 10cm vorgegeben, mit einer zulässigen Masse von 1,33 kg. Für größere Volumen und Massen können mehrere Einheiten von CubeSats verbunden werden. Satelliten mit 1U, 1.5U, 2U, oder 3U können von dem einheitlichen Startmechanismus (P-Pod) in die Erdumlaufbahn ausgelassen werden. Die Kosten von CubeSat Missionen können gering gehalten werden, indem diese als sekundäre Nutzlast bei Raketenstarts mitfliegen. Für größere Satelliten (6U, 12U, 27U) werden andere Startmechanismen benötigt. Die zugelassene Masse wird jedoch auf 2 kg/U angehoben. Allgemein gelten noch weitere Vorschriften für die verwendeten  Materialien, Kommunikationsfähigkeit, gespeicherte Energie und Aktivierungszeitpunkt der Systeme nach Einsatz in die Umlaufbahn.
Falls ein Entwurf nicht den Vorschriften entspricht, kann bei dem Betreiber der Trägerrakete eine Sondergenehmigung angefragt werden. Nach einer Reihe von Tests entscheidet dieser ob er die Abweichungen akzeptiert, Änderungen vorgenommen werden müssen, oder ein anderer Anbieter gefunden werden muss. 
	\end{flushleft}

			
	\section{Cubesat Subsysteme}
	hier Hauptsächlich das Fazit von max kompakt darstellen und refernzieren
		\subsection{Antrieb - propulsion}
		\subsection{Energie - EPS}
		\subsection{Guidance, navigation and control -GNC ADCS}
		\subsection{Command and data handling}
		\subsection{Communications}
		\subsection{Thermal}
		\subsection{Structure}
				
	\section{RDVDO Mechanismen} das kann ausführlich sein
		\subsection{Docking Strategien}
						\textbf{Roboterarm}
						\textbf{Fangnetz}
						\textbf{Adhäsiv Docken}
						Übersichtstabelle/Graphik: sihe die Quellen die ich am 15.05.2019 gezeigt habe
		\subsection{Bionische Materialien}
						\textbf{Was sind Geckomaterialen}
						\textbf{Bisher getestete Gecko-Materialien}
						\textbf{Bisherige Erfolge}
						\textbf{State of the Art}
						\textbf{Problematik}