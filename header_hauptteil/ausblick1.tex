\chapter{Fazit und Ausblick}
	\hfill\emph{(Florian Czorny, Valentina Dietrich, Oussama Mouhaya, Frederik Schäfer, Marc Strempel)}	\\
Wie die Arbeit gezeigt hat, kann mit Einschränkungen angenommen werden, dass das vorgeschlagene CubeSate Design in der Lage ist Starlink ähnliche Ziele aus Höhen von \num{400} bis \SI{1400}{\km} aktiv wieder eintreten zu lassen. Bei den GMAT-Simulationen wurde angenommen, dass durchgehend die maximale Leistung zu jedem beliebigen Zeitpunkt verfügbar ist. Dabei wurde auch der Schatten der Erde nicht Berücksichtigt. Mit Einbeziehung dieser Faktoren kann sich die benötigte Zeit verändern. Bei der Budgetierung mit QuSad wurden teilweise nur angenäherte Werte verwendet. Einige Werte für die verwendeten Systeme wurden auf Basis von existierenden Systemen abgeleitet und skaliert. Daraus kann eine Veränderung der Massen und des Volumens erfolgen. Über sämtliche Komponenten bei denen eine Skalierung stattgefunden hat, sollte Rücksprache mit den Herstellern gehalten werden, um möglichst reale Werte für die Berechnungen zur Verfügung zu haben.  Veränderungen der Masse aufgrund des Treibstoffverbrauchs, sowie die Verringerung der Umlaufbahn  wird nicht berücksichtigt. Eine Rückkopplung beider Programme fand statt, indem eine Konfiguration mit beiden Programmen getestet wurde. Zu beachten ist, dass die genannten Vereinfachungen in Kombination für weitere Abweichungen sorgen. Aus diesem Grund sind die Ergebnisse nur als Richtwerte anzusehen und sollten bei Bedarf mit einer weiterführenden Analyse genauer betrachtet werden. Mit Hilfe der Simulationen konnte gezeigt werden, dass mit dem selben Aufbau auch deutlich größere Massen als die der bekannten Megakonstellationssatelliten abgebremst werden können. Selbst bei einer Höhe von \SI{1400}{\km} verlief die Simulation mit \SI{475}{\kilogram} noch erfolgreich. Die derzeit größte bekannte Masse geplanter Megakonstellationssatelliten liegt mit \SI{386}{\kilogram} deutlich unter diesem Wert \cite{BenLarbi.2017}. Bei der simulierten Konfiguration war die Treibstoffmenge (\SI{10}{\kilogram}) der limitierende Faktor. Durch eine Erhöhung der Treibstoffmasse um \num{50}\% könnte die Zielgruppe auf größere Massen in höheren Umlaufbahnen erweitert werden. Dies lässt sich unter Berücksichtigung des tatsächlichen CubeSat-Volumens von \SI{42840}{\cubic\cm} laut der erstellten Budgets realisieren.  Zielsetzung der vorliegenden Arbeit war die Eignung des angenommen CubeSat Designs für ADR Missionen zu testen. Da in den Simulationen die selbst gesetzten Ziele erreicht worden sind, benötigt das gewählte Design keine grundlegende Veränderung. Anhand der Ergebnisse scheint die Durchführbarkeit einer ADR Mission mit CubeSats realisierbar zu sein. Anknüpfend an diese Ergebnisse scheint es sinnvoll den Fokus nun auf RDV und Docking, in Verbindung mit dem RCS und der Sensorik, zu lenken.