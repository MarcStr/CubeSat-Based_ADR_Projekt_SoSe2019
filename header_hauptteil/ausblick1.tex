\chapter{Fazit und Ausblick}
Zusammenfassend lässt sich sagen, dass die Simulation der Missionen sehr gut verlaufen ist. Über alle betrachteten Höhen lassen sich, laut der Simulationsergebnisse, defekte Satelliten in der Größenordnung von Megakonstellationen aus der Umlaufbahn entfernen. Entgegen der Erwartungen scheint es sogar möglich zu sein mit dem selben Aufbau deutlich größere Massen abzubremsen. Selbst bei einer Höhe von \SI{1400}{\kilo\metre} verlief die Simulation mit \SI{475}{\kilogram} noch erfolgreich. Die derzeit größte bekannte Masse geplanter Megakonstellationssatelliten liegt mit \SI{386}{\kilogram} deutlich unter diesem Wert \cite{BenLarbi.2017}. Der limitierende Faktor der Mission ist die Treibstoffmenge von \SI{10}{\kilogram}. Eine Testsimulation hat ergeben, dass die Zielgruppe auf größere Massen in hohen Umlaufbahnen erweitert werden kann, wenn die Treibstoffmenge um \num{50}\% erhöht wird. Das wäre zwar innerhalb des Massenbudgets, jedoch ist das Volumen mit der ursprünglichen Auslegung bereits komplett ausgeschöpft. Die Ursache dafür sind unter anderem die Skalierungen der Subsysteme. Des Weiteren müssen die Einbauvorrichtungen der einzelnen Komponenten mit berücksichtigt werden, die bei den Budgets durch das Programm nicht beachtet werden konnten. Zielsetzung der vorliegenden Arbeit war die Eignung des angenommen CubeSat Designs für ADR Missionen zu testen. Da die  Simulationen erfolgreich verlaufen sind benötigt das gewählte Design keine Optimierung. Dies bildet eine gute Voraussetzung für die weiteren Planungen bezüglich der Satellitenkonstellation. Mit diesem Nachweis kann jetzt an dem Wärmemanagement, RCS und Struktur gearbeitet werden, die in dieser Arbeit nur untergeordnet behandelt wurden. Weiterhin wurde festgestellt, dass keine genaue Aussage über die Effizienz des RCS getroffen werden kann, da die RCS-Treibwerksauswahl auf Skalierung basiert. Über sämtliche Komponenten bei denen eine Skalierung stattgefunden hat sollte Rücksprache mit den Herstellern gehalten werden, um möglichst reale Werte für die Berechnungen zur Verfügung zu haben. 
