\chapter{Einleitung}
		\section{Motivation}
Seitdem ersten Satelliten Sputnik 1, welcher am 04. Oktober 1957 in die Erdumlaufbahn geschossen wurde, kommen immer mehr Satelliten hinzu. Diese bilden die Grundlage unserer modernen Kommunikation und sind aus dem menschlichen Alltag nicht wegzudenken. Jedoch ist der Platz in den von Menschen nutzbaren Umlaufbahnen um die Erde begrenzt. Neben den bereits vorhandenen Satelliten werden jedes Jahr neue in den Erdorbit befördert. Beim Start von Satelliten und Sonden bleiben Raketenstufen in der Erdumlaufbahn zurück, welche aufgrund ihrer Manövrierunfähigkeit eine Gefahr für Menschen und Maschinen im Weltall bilden. Moderne Low-Earth-Orbit (LEO)-Satelliten werden so entwickelt, dass diese am Ende ihrer Lebensspanne wieder in die Erdatmosphäre eintreten und verglühen. 
Sollten diese Objekte eine Funktionsstörung haben oder beschädigt sein, können diese den selbst initiiert Wiedereintritt nicht durchführen. Da diese Objekte unkontrolliert im Weltraum treiben, besteht die Möglichkeit einer Kollision mit anderen Satelliten. Dabei zersplittern die Trümmer in einem Kaskadeneffekt und weitere Kollisionen sind unumgänglich. Die dabei entstehenden Kleinstteile haben ein sehr hohes Gefahrenpotential, da diese mit aktuellen Mitteln nur begrenzt detektiert werden können. 
Die Gesamtheit aller nicht funktionsfähigen Objekten, dazu zählen defekte Satelliten sowie Trümmer und Kleinstteile, werden als Weltraummüll bezeichnet. Um das Risiko der Kollision von Satelliten mit Weltraummüll zu vermindern, muss dieser aktiv entfernt werden. Dazu wird im Folgenden auf die Entfernung des Weltraummülls mit Nanosatelliten eingegangen.

		\section{Problemstellung}
Ohne zuverlässige ADR Maßnahmen steigt die Wahrscheinlichkeit erneuter Kollisionen, die in einer Kaskade an Trümmerteilen enden können. Um dieser Entwicklung entgegenzuwirken werden verschiedene ADR Methoden untersucht.
Im Folgenden wird darauf eingegangen ob CubeSats eine realisierbare Plattform für ADR Missionen darstellen.
Auf die geeignete Auswahl an Subsystemen muss besonderer Wert gelegt werden. Grund dafür ist die Einschränkung in Masse und Volumen, welche direkten Einfluss auf das Energiebudget haben. 
Da die Zielobjekte eine angenommene Masse von bis zu 400 kg haben, ist es notwendig eine geeignete Konfiguration auszuwählen. Im Fokus steht die Auswahl des Triebwerks, da die CubeSats nach dem Docking ein Vielfaches des Eigengewichtes bewegen müssen. 
Um eine bessere Betrachtung der Problematik zu ermöglichen wird angenommen, dass der CubeSat sich mit einem Abstand von 10 km auf der gleichen Umlaufbahn wie das Zielobjekt befindet. Ein essentielles Problem bei einer ADR Mission ist das Docking, da hohe Anforderungen an die Lageregelung, Navigation und Positionsbestimmung gestellt sind.

		\section{Struktur}
		
In Kapitel 2 werden die einzelnen Subsysteme von Satelliten aufgeführt und im Allgemeinen beschrieben.
Kapitel 3 befasst sich mit der Auswahl von Komponenten und dem Vorschlag von einem CubeSat Design. [QUSAD Team]
Das vierte Kapitel beschreibt die Missionen und wie diese mit Hilfe des General Mission Analysis Tool (GMAT) simuliert werden. Im Anschluss werden die Simulationergebnisse aufgeführt und beschrieben. 

		\section{Stand der Technik}
Auf Grund der Vielseitigkeit von CubeSat Anwendung ist der Stand der Technik im ständigen Wandel. Dennoch sind die einzelnen Subsysteme durch ein hohen Technology Readyness Level (TRL) ausgezeichnet, was daran liegt, dass viele bereits etablierte Technologien lediglich herunterskaliert werden müssen.
Bisher dienten viele Missionen hauptsächlich zur Demonstration der einzelnen Bauteile. Ein weiterer Grund für das hohe TRL ist die Tatsache, dass die meisten Subsysteme bereits auf dem kommerziellen Markt erhältlich sind.
Einige Subsysteme werden zuerst auf größeren Satelliten auf die Probe gestellt. Zum Beispiel zeigte die AVANTI Mission, dass es möglich ist Rendezvous nur über Kameras zu vollziehen, die auch für einen CubeSat nutzbar sind \cite{Gaias.2018}, \cite{Gaias.2018b}. Ein weiteres Beispiel für Proof of Concept, in diesem Fall von einem Electrical Power System (EPS), ist der Aoxiang-Sat \cite{Peng.2018}. Mit diesem 12U CubeSat wurde ein neuartiges EPS für Nanosatelliten erprobt. Zusätzlich wurde zum ersten Mal von einem 12U CubeSat von der Erdatmosphäre polarisiertes Licht gemessen.
Neben den vielen Experimenten im LEO gibt es geplante Mond Missionen, wie LunarCube \cite{}. Auch im interplanetaren Raum sind bereits CubeSats unterwegs. Die Mission Mars Cube One (MarCO) wurde 2018 zusammen mit der InSight Landesonde gestartet und zeigt, dass CubeSats auch für interplanetare Missionen geeignet sind \cite{}.


	% Overview of cooperative RDVDO missions\\
	%	Overview of uncooperative RDVDO mission\\
