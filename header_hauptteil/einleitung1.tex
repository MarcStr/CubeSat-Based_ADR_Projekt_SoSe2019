\chapter{Einleitung}
		\section{Motivation}
	\hfill\emph{(Oussama Mouhaya)}\\
Seit dem ersten Satelliten Sputnik \num{1}, welcher am 04. Oktober 1957 in die Erdumlaufbahn geschossen wurde, kommen immer mehr Satelliten hinzu. Diese bilden die Grundlage unserer modernen Kommunikation und sind aus dem menschlichen Alltag nicht wegzudenken. Jedoch ist der Platz in den von Menschen nutzbaren Umlaufbahnen um die Erde begrenzt. Neben den bereits vorhandenen Satelliten werden jedes Jahr neue in den Erdorbit befördert. Beim Start von Satelliten und Sonden bleiben Raketenstufen in der Erdumlaufbahn zurück, welche aufgrund ihrer Manövrierunfähigkeit eine Gefahr für Menschen und Maschinen im Weltall bilden. Moderne Low-Earth-Orbit (LEO)-Satelliten werden so entwickelt, dass diese am Ende ihrer Lebensspanne wieder in die Erdatmosphäre eintreten und verglühen\cite{.e}. Sollten diese Objekte eine Funktionsstörung haben oder beschädigt sein, können diese den selbst initiiert Wiedereintritt nicht durchführen. Da diese Objekte unkontrolliert im Weltraum treiben, besteht die Möglichkeit einer Kollision mit anderen Satelliten. Dabei zersplittern die Trümmer in einem Kaskadeneffekt und weitere Kollisionen sind unumgänglich. Die dabei entstehenden Kleinstteile haben ein sehr hohes Gefahrenpotential, da diese mit aktuellen Mitteln nur begrenzt detektiert werden können. Die Gesamtheit aller nicht funktionsfähigen Objekte, dazu zählen defekte Satelliten sowie Trümmer und Kleinstteile, werden als Weltraummüll bezeichnet. Um das Risiko der Kollision von Satelliten mit Weltraummüll zu vermindern, muss dieser aktiv entfernt werden. Dazu wird im Folgenden auf die Entfernung des Weltraummülls mit Nanosatelliten eingegangen.

		\section{Problemstellung}
		\hfill\emph{(Florian Czorny)}\\
Ohne zuverlässige ADR Maßnahmen steigt die Wahrscheinlichkeit erneuter Kollisionen, die in einer Kaskade an Trümmerteilen enden können. Um dieser Entwicklung entgegenzuwirken werden verschiedene ADR Methoden untersucht.
Im Folgenden wird darauf eingegangen ob CubeSats eine realisierbare Plattform für ADR Missionen darstellen.
Auf die geeignete Auswahl an Subsystemen muss besonderer Wert gelegt werden. Grund dafür ist die Einschränkung in Masse und Volumen, welche direkten Einfluss auf das Energiebudget haben. Da die Zielobjekte eine angenommene Masse von bis zu \SI{500}{\kilogram} haben, ist es notwendig eine geeignete Konfiguration auszuwählen. Im Fokus steht die Auswahl des Triebwerks, da die CubeSats nach dem Docking ein Vielfaches des Eigengewichtes bewegen müssen. Um eine bessere Betrachtung der Problematik zu ermöglichen wird angenommen, dass der CubeSat sich mit einem Abstand von \SI{10}{\kilo\metre} auf der gleichen Umlaufbahn wie das Zielobjekt befindet. Ein essentielles Problem bei einer ADR Mission ist das Docking, da hohe Anforderungen an die Lageregelung, Navigation und Positionsbestimmung gestellt sind. 

		\section{Stand der Technik}
		\hfill\emph{(Marc Strempel, Frederik Schäfer)}\\	
Das aktive Entfernen von Weltraumschrott mit spezialisierten Satelliten wurde bislang noch nicht durchgeführt. Es gab jedoch einige Versuche, die sich verschiedenen Aspekten dieser Herausforderung befasst haben. So wurden z.B. bei der AVANTI-Mission des DLR Rendezvous Manöver mit einem nicht-kooperativen Ziel durchgeführt \cite{Gaias.2018,Gaias.2018b}. 
Außerdem wurden unterschiedliche Konzepte untersucht, mit denen sich ein spezialisierter Satellit mit einem defekten verbinden lässt (z.B. Roboterarm, Fangnetz, etc.[Kap.\ref{ADRm}]). Diese befinden sich allerdings noch in der Konzeptphase\cite{Mark.2019}.

Nanosatelliten nach CubeSat Standard werden, aktuell häufig für Proof of Concept Missionen verwendet, da die Start- und Konstruktionskosten im Vergleich zu größeren Satelliten gering ausfallen. Eine dieser Missionen ist der Axiong Satellit, welcher die Polarisierung des Lichts in der Erdatmosphäre messen konnte und dabei ein neuartiges EPS für CubeSats getestet hat [cite axiong]. Ein weiteres Beispiel ist eine Studie, bei der überprüft wird, ob ein CubeSat mit Hall-Effekt Triebwerk Untersuchungen an NEOs durchführen kann \cite{UniversityofStrathclydeGlasgow.2018}. Die Verwendung von CubeSats für solche PoC-Missionen ist auch einer der Gründe weshalb das TRL vieler CubeSat Bauteile relativ hoch ist.



Die Idee CubeSats für ADR-Missionen einzusetzen wurde unter Anderem von der University of Strathclyde in Simulationen getestet. Dabei wurde ersichtlich, dass es prinzipiell möglich ist CubeSats mit der notwendigen Technik für eine solche Mission auszustatten. Der Fokus der Simulation lag allerdings auf der Funktion der GNC Software und nicht auf dem Wiedereintrittsvorgang\cite{Pirat.2017}. Eine weitere Studie stellt CubeSats als mögliche Plattform für ADR Missionen dar, deren Hauptzielgruppe Satelliten zukünftiger Megakonstellationen sind\cite{BenLarbi.2017}.


	\section{Methodik und Aufbau der Arbeit}
		\hfill\emph{(Valentina Dietrich)}\\	
Im Rahmen dieser Arbeit soll eine  CubeSat basierte ADR-Mission untersucht werden. Insbesondere werden das CubeSat- und Missionsdesign analysiert und ausgewertet. Zunächst werden in \kap{theorie} die einzelnen Subsysteme von Satelliten aufgeführt und beschrieben. Zusätzlich werden der Rendezvous- und Docking Vorgang näher erklärt. Im darauffolgenden Kapitel wird eine Designanalyse mittels Budgetplanung für ein bestehendes CubeSat Design durchgeführt.  Das vierte Kapitel beschäftigt sich mit der Missionsplanung und unter Zuhilfenahme des General Mission Analysis Tool (GMAT) werden Missionsabläufe simuliert. Abschließend werden die Simulationsergebnisse ausgewertet, um die Eignung des vorher bestimmten Satellitenentwurfes zu ermitteln.
	% Overview of cooperative RDVDO missions\\
	%	Overview of uncooperative RDVDO mission\\
