\chapter{Auswertung des CubeSat-based ADR Konzepts}

\section{Bewertungsstrategie}
	Die Strategie besteht draus mehrere Simulationen per GMAT für untershciedliche Koonfigs durchzuführen. Die Bewertung fokusiertsich auf die Machbarkeit des De-orbiting (seihe TODOS.txt)\\

		\subsection{Kriterien der Bewertung}
					
		\subsection{GMAT}


\section{Ergebnisse}
	
		\subsection {Generated Data}
For every considered satellite design (=mainly thruster configuration, 3-4 different designs), generate following data: \\

Deorbit time and spent fuel mass for all:
\begin{itemize}
	\item Masses from 50-500 kg\\
	\item Altitudes from 1400-400? km  (ggf. semi-major axis) \\
	\item Eccentricities from ~0 to highest recorded eccentricity of debris in <1400 km orbit
\end{itemize}

Note: Output EVERY relevant simulation parameter (initial orbit and S/C data, burn start/stop angles, start epoch etc.)  at the beginning of every simulation run [discuss with Max]
	
			\subsection{Reachability Enveloppe}
RESULTING DIAGRAMS: \\

\begin{enumerate}
		\item Visualize the absolute performance of the main design (Max), e = 0 \\
				\begin{itemize}
						\item Axes: y = mass, x = SMA \\
						\item Graph: Use color gradients to display deorbit times (same time = same color)
				\end{itemize}
				\item  Visualize the influence of eccentricity on deorbit times using the main design \\
				\begin{itemize}
						\item Axes: y = mass, x = SMA \\
						\item Select a fixed deorbit time (e.g. 2 years) \\
						\item Graph: Use color gradients to display eccentricities (same ecc = same color)
				\end{itemize}
				\item Visually compare the performances of the different designs (Max \& 2-3 group designs), e = 0 
				\begin{itemize}
						\item Axes: y = mass, x = SMA \\
						\item Select 1-2 fixed deorbit times (e.g. 2 years \& 5 years) \\
						\item Graph: Draw lines of same deorbit time (selected above) for each of the different designs
				\end{itemize}				
\end{enumerate}
Optional for group after 3. (decide if worth it)\\
4. Repeat 1. with all other chosen designs\\

NOTES: \\

For 1. \& 4.:\\
(Deorbit time limited to 10 years (15? 20?)) \\
-> <2 years of deorbiting takes 3-4 mins to simulate, amount of data is immense => limit maximum deorbit time? \\
-> At which deorbiting time does a feasible solution become unattractive? If 200 kg in 1000 km orbit is \\
   deorbitable in 20 years, is it worth it? Better use chemical deorbiting = different mission in this case? \\
---> Agree on a meaningful limit on deorbit time \\

For all graphs/simulations: \\
Fuel mass limited by design -> if 27U standard is to be kept no matter which thruster configuration is \\
chosen, then smaller/more lightweight thrusters would result in more available space for fuel \\
---> Agree on a set percentage of margin for all designs (e.g. 50\%) and determine maximum fuel from there? \\

-> For Max's design, fuel mass limited to 10 kg (thinking about increasing to 15 kg)\\
			%\subsection{Missionsauslegung}
					%\textbf{Mögliche CubeSat Konfigurationen}\\
			%\subsection{Missionssimulation}
					%\textbf{Impuls-basierter Schub}\\
					%\textbf{Begrenzter Schub}\\
					%\textbf{Mission mit Randbedingungen}\\
			%\subsection{Konfigurationsvergleich}
					%\textbf{..}\\
					%\textbf{..}\\
					%\textbf{..}\\