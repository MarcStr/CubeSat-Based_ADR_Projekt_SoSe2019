\chapter{Projektmanagement}
\label{cha:projekt}

\section{Work Breakdown Structure}
\label{sec:wbs}

\begin{landscape}
\begin{tikzpicture}[
  basic/.style   = {draw, text width=2.7cm, align=left, drop shadow, rectangle},
  root/.style    = {basic, text width=12cm, rounded corners=2pt, thin, align=center, fill=gray90},
  level 2/.style = {basic, rounded corners=2pt, thin, fill=gray80},
  level 3/.style = {basic, thin, fill=gray90, text width=2.6cm},
  level 1/.style={sibling distance=38mm}, edge from parent fork down, 
  edge from parent/.style={->,draw}, level distance=2.5cm,  >=latex]

% root of the the initial tree, level 1
\node[root] {\tb{Analyse einer CubeSat basierenden ADR-Mission}}
% The first level, as children of the initial tree
  child {node[level 2] (c1) {\tb{AP~1000} \\ Literatur-recherche}}
  child {node[level 2] (c2) {\tb{AP~2000} \\ CubeSat Design}}
  child {node[level 2] (c3) {\tb{AP~3000} \\ Budgetplanung mit QuSad}}
  child {node[level 2] (c4) {\tb{AP~4000} \\ Missionsplanung mit GMat}}
  child {node[level 2] (c5) {\tb{AP~5000} \\ Dokumentation \\ $~~$}}

% The second level, relatively positioned nodes
\begin{scope}[every node/.style={level 3}, node distance=6pt]
\node [below=of c1, xshift=10pt] (c11) {\tb{AP~1100} \\ Weltraumm\"ull-problematik};
\node [below=of c11] (c12) {\tb{AP~1200} \\ CubeSat Subsysteme};
\node [below=of c12] (c13) {\tb{AP~1300} \\ Rendezvou und Docking};
\node [below=of c13] (c14) {\tb{AP~1300} \\ Erforderliche Softwares};


\node [below=of c2, xshift=10pt] (c21) {\tb{AP~2100} \\ CubeSat Datenbank Analyse};
\node [below=of c21] (c22) {\tb{AP~2200} \\ Ausarbeitung von effizienten Subsystemen};
\node [below=of c22] (c23) {\tb{AP~2300} \\ Ausarbeitung von CubeSat Konfigurationen};

\node [below=of c3, xshift=10pt] (c31) {\tb{AP~3100} \\ Datenbankerweiterung};
\node [below=of c31] (c32) {\tb{AP~3200} \\ Generierung der CubeSat Konfigurationen};
\node [below=of c32] (c33) {\tb{AP~3300} \\ Budgetkalkulation};
\node [below=of c33] (c34) {\tb{AP~3300} \\ Gegenüberstellung der Konfigurationen};

\node [below=of c4, xshift=10pt] (c41) {\tb{AP~4100} \\ Missionsauslegung};
\node [below=of c41] (c42) {\tb{AP~4200} \\ Durchführung der Simulation};
\node [below=of c42] (c43) {\tb{AP~4200} \\ Auswertung der Simulation};

\node [below=of c5, xshift=10pt] (c51) {\tb{AP~5100} \\ Textverfassung};
\node [below=of c51] (c52) {\tb{AP~5200} \\ Überprüfung};

\end{scope}

% lines from each level 1 node to every one of its "children"
\foreach \value in {1,2,3}
  \draw[->] (c1.212) |- (c1\value.west);

\foreach \value in {1,2,3}
  \draw[->] (c2.212) |- (c2\value.west);

\foreach \value in {1,2,3}
  \draw[->] (c3.212) |- (c3\value.west);

\foreach \value in {1,2}
  \draw[->] (c4.212) |- (c4\value.west);

\foreach \value in {1,2}
  \draw[->] (c5.212) |- (c5\value.west);
  
\end{tikzpicture}
\end{landscape}

\section{Zeitplan}
\label{sec:zeitplan}

\begin{landscape}
%\noindent\resizebox{\textwidth}{!}{	% Einfügen, falls zu groß
\begin{ganttchart}[hgrid,
                   time slot format = isodate, 
                   x unit=0.28cm,	% Zum komprimieren des Charts in x-Richtung
                   %y unit chart=0.7cm,
                   %compress calendar,	% Komprimiert den Chart in der Breite
                   calendar week text = {Woche~\currentweek},
                   chart element start border = right,
                   bar/.append style={fill=blue!40, rounded corners=2pt},
                   bar incomplete/.append style={fill=blue!10},
                   bar label node/.append style={align=left, text width=7cm},
                   group label node/.append style={align=left, text width=8cm},
                   milestone label node/.append style={align=left, text width=8cm},
                   bar progress label node/.style={right=2mm},
                   progress label text = {\pgfmathprintnumber[precision=0, verbatim]{#1}\%},
                  ]{2013-01-01}{2013-02-23}
  \gantttitlecalendar{year, month=shortname, week}\\
  %\gantttitle{2013}{59}\\
  \ganttgroup{AP 1000: Literaturrecherche}{2013-01-01}{2013-02-01}\\
  \ganttbar[progress=50]  {AP 1100: Weltraummüllproblematik}{2013-01-01}{2013-01-08}\\
  \ganttlinkedbar {AP 1200: Beobachtung von Weltraummüll}{2013-01-09}{2013-01-17}\\
  \ganttlinkedbar {AP 1300: Bahnbestimmung mit optischen Sensoren}{2013-01-17}{2013-02-01}\\
  %\ganttbar[progress=100]{AP 1300: TEXT}{2013-01-01}{2013-01-30}\\	% Beispiel für Fortschrittsbalken!
  
  \ganttgroup{AP 2000: Algorithmuserstellung}{2013-02-02}{2013-02-23}\\
  \ganttbar  {AP 2100: ...}{2013-02-02}{2013-02-09}\\
  \ganttbar  {AP 2200: ...}{2013-02-10}{2013-02-19}\\
  
  \ganttmilestone{Meilenstein}{2013-02-20}\\
  
 \end{ganttchart}
%}
\end{landscape}

\section{Work Package Description}
\label{sec:wpd}

\begin{table}[!h]
 \begin{center}
  \begin{tabular}{|p{35mm}||p{55mm}|p{50mm}||p{40mm}|}
   \hline
   \multicolumn{3}{|l||}{\textbf{}} & \multicolumn{1}{c|}{}\\
   \multicolumn{3}{|l||}{\textbf{}} & \multicolumn{1}{c|}{\textbf{AP 1100}}\\
   \multicolumn{3}{|l||}{\textbf{}} & \multicolumn{1}{c|}{}\\
   \hline\hline
   \textbf{Titel} & \multicolumn{2}{p{7cm}||}{\textbf{Genauigkeit der Bahnbestimmung von Space Debris-Objekten mittels weltraumgestützter optischer Sensoren}} 
& \textbf{Seite:} 1 von 1\\
   \hline
   \textbf{Verantwortlicher} & \multicolumn{2}{l||}{Sebastian Stabroth} & \textbf{Version:} 1.0\\
   \hline
   \multicolumn{3}{|l||}{} & \textbf{Datum:} 25.06.2003\\
   \hline\hline
   \textbf{Beginn} & \multicolumn{2}{l||}{T$_0$} & \\
   \hline
   \textbf{Ende} & \multicolumn{2}{l||}{T$_0$+1 Woche} & \textbf{Dauer}: 1 Woche\\
   \hline\hline
   \textbf{Bearbeiter} & \multicolumn{3}{l|}{Sebastian Stabroth}\\
   \hline\hline
   \multicolumn{4}{|p{150mm}|}{\textbf{Ziele:}}\\
   \multicolumn{4}{|p{150mm}|}{$\bullet$ Kenntnis über die Weltraummüllumgebung, Bahnbereiche von Space Debris-Populationen, Objektanzahlen und -größen}\\
   \multicolumn{4}{|p{150mm}|}{}\\
   \multicolumn{4}{|p{150mm}|}{\textbf{Input:}}\\
   \multicolumn{4}{|p{150mm}|}{$\bullet$ Literatur zum Thema Weltraummüll}\\
   \multicolumn{4}{|p{150mm}|}{}\\
   \multicolumn{4}{|p{150mm}|}{\textbf{Schnittstellen zu anderen APs:}}\\
   \multicolumn{4}{|p{150mm}|}{$\bullet$ \textbf{AP~5100} zur Simulation der Bahnbestimmung von Weltraummüll}\\
   \multicolumn{4}{|p{150mm}|}{}\\
   \multicolumn{4}{|p{150mm}|}{\textbf{Aufgaben:}}\\
   \multicolumn{4}{|p{150mm}|}{$\bullet$ Einlesen in die Thematik Weltraummüll}\\
   \multicolumn{4}{|p{150mm}|}{}\\
   \multicolumn{4}{|p{150mm}|}{\textbf{Ergebnisse:}}\\
   \multicolumn{4}{|p{150mm}|}{$\bullet$ Verständnis der Weltraummüllproblematik}\\
   \hline
  \end{tabular}
 \end{center}
\end{table}

\clearpage

\begin{table}[!h]
 \begin{center}
  \begin{tabular}{|p{35mm}||p{55mm}|p{50mm}||p{40mm}|}
   \hline
   \multicolumn{3}{|l||}{\textbf{}} & \multicolumn{1}{c|}{}\\
   \multicolumn{3}{|l||}{\textbf{}} & \multicolumn{1}{c|}{\textbf{AP 1200}}\\
   \multicolumn{3}{|l||}{\textbf{}} & \multicolumn{1}{c|}{}\\
   \hline\hline
   \textbf{Titel} & \multicolumn{2}{p{7cm}||}{\textbf{Titel des Arbeitspakets}} & \textbf{Seite:} X von Y\\
   \hline
   \textbf{Verantwortlicher} & \multicolumn{2}{l||}{Dein Name} & \textbf{Version:} 1.1\\
   \hline
   \multicolumn{3}{|l||}{} & \textbf{Datum:} DD.MM.YYYY\\
   \hline\hline
   \textbf{Beginn} & \multicolumn{2}{l||}{T$_0$} & \\
   \hline
   \textbf{Ende} & \multicolumn{2}{l||}{T$_0$+X Wochen} & \textbf{Dauer}: X Wochen\\
   \hline\hline
   \textbf{Bearbeiter} & \multicolumn{3}{l|}{Dein Name}\\
   \hline\hline
   \multicolumn{4}{|p{150mm}|}{\textbf{Ziele:}}\\
   \multicolumn{4}{|p{150mm}|}{$\bullet$ Ziel 1}\\
   \multicolumn{4}{|p{150mm}|}{$\bullet$ Ziel 2}\\
   \multicolumn{4}{|p{150mm}|}{$\bullet$ ...}\\
   \multicolumn{4}{|p{150mm}|}{}\\
   \multicolumn{4}{|p{150mm}|}{\textbf{Input:}}\\
   \multicolumn{4}{|p{150mm}|}{$\bullet$ Input 1}\\
   \multicolumn{4}{|p{150mm}|}{$\bullet$ ...}\\
   \multicolumn{4}{|p{150mm}|}{}\\
   \multicolumn{4}{|p{150mm}|}{\textbf{Schnittstellen zu anderen APs:}}\\
   \multicolumn{4}{|p{150mm}|}{$\bullet$ \textbf{AP XXXX} Beschreibung}\\
   \multicolumn{4}{|p{150mm}|}{$\bullet$ \textbf{AP ....} ...}\\
   \multicolumn{4}{|p{150mm}|}{}\\
   \multicolumn{4}{|p{150mm}|}{\textbf{Aufgaben:}}\\
   \multicolumn{4}{|p{150mm}|}{$\bullet$ Aufgabe 1}\\
   \multicolumn{4}{|p{150mm}|}{$\bullet$ ...}\\
   \multicolumn{4}{|p{150mm}|}{}\\
   \multicolumn{4}{|p{150mm}|}{\textbf{Ergebnisse:}}\\
   \multicolumn{4}{|p{150mm}|}{$\bullet$ Ergebnis 1}\\
   \multicolumn{4}{|p{150mm}|}{$\bullet$ ...}\\
   \hline
  \end{tabular}
 \end{center}
\end{table}